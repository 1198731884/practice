\section{Introduction}
\subsection{Background}
\noindent
Now more and more choices of outdoor travel, especially the selection of water drifting, are of great importance for how to solve the problem of optimal utilization of camping sites and not exceeding the maximum river carrying capacity along the "Great River" camping. At present, the plan of river transport administration is not reasonable. The planning of travel time and type of travel team is static. There are numerous choices for tourists. There are more occupation conflicts in campsites on river banks, and the utilization rate of rivers and camps is not utilized. Maximum. We transform the practical problem into a mathematical model, and use dynamic programming, task assignment model and matrix decomposition to propose to maximize the use of campsites under the maximum carrying capacity of rivers and reduce the number of encounters between drifting ships, as compared with the original number of traveling teams.
\subsection{Restatement of the problem}
\noindent
When travellers drift on a long river of 225 miles, they can enter at the only entrance and leave the only exit. Passengers take either oar-powered rubber rafts, which travel on average 4 mph or motorized boats, which travel on average 8 mph. The journey lasts from 6 to 18 nights, during which time the tourists can camp on the river bank and are deemed to be evenly distributed along the river corridor. Due to weather reasons, the river opens to people only six months a year (assuming 30 days per month), During this six-month period, there are altogether x trips.And there are Y camp sites on the Big Long River.
\par Now the administration hires us to provide a new plan. This programme requires that the number of travel tours (river carrying capacity) be maximized to the maximum extent possible, and that there should be a minimum number of contacts among the tour teams, and it is noteworthy that a camp site can accommodate only one tour group per night.
\subsection{Our works}
\noindent
We assume that the behavior of travelers is random and that their choice of travel time, means of transport, and campgrounds is in accordance with a certain probability distribution. In order to determine the parameters of the probability distribution, we first determine the mean and variance of the three alternatives.
\par We assume that the behavior of travelers is random and that their choice of travel time, means of transport, and campgrounds is in accordance with a certain probability distribution.And we think that once a tour group chooses the means oftransport (oar- rafts or motorboat), it will not change throughout the journey. Then we consider the type of travel time and mode of promotion, the type of trip plan is divided into 25 kinds, respectively, recorded as ${p_1},{p_2},{p_3},...,{p_{25}}$ ,Let $i$ denote the number of days camping in the travel plan,where $i = 6,7,...,18$ . After that, we define the unit camping matrix $U_i^k$ , camping matrix ${C_i}$, total camping matrix $A$, and departure date matrix ${D_i}$. Among them,the unit camping matrix ($U_i^k$$k = 1,2,3,...$) reflects the occupation of thecamping sites along the river bank when the trip plan ${p_i}$ is executed on the $k$ th time, it is a ${n_i} \times Y$ matrix, The camping matrix ${C_i}$ reflects the case of travel plans ${p_i}$ occupying camps or not throughout the rafting season. We aggregate the matrix $U_i^k$ to get a matrix of $(i \times k) \times Y$ and then we add the zero matrix of $1 \times Y$  , and these zero matrices correspond to the night when travel plan ${p_i}$ is not executed, so we get a matrix of $179 \times Y$ and define it as ${C_i}$(there are 25 such matrices in total).Total Camp matrix shows the utilization of the campsite throughout the drifting season, which is the sum of the matrix ${C_i}$ and has a dimension of $179 \times Y$ the departure date matrix $A$ records some cases of the travel team within 180 days.      
%\par First of all, according to the subject of a total of $X$ trip throughout the rafting season, there are $Y$ campsites evenly distributed along the riverbank. As assumed, we think the entire drift season lasts for a total of 180 days, including 179 nights, and we use the natural number $1,2,3,...,179$ to mark the 179 nights. And we think that once a group chooses to transport a vehicle (a dinghy or a motorboat). It will not change throughout the journey. Then we will consider the type of journey time and classify the types of the journey into 25 kinds, namely ${p_1},{p_2},{p_3},...,{p_{25}}$ $i$ indicates the number of days in the travel plan $i = 6,7,...,18$. After that, we define unit camping matrix $U_i^k$, camping matrix ${C_i}$, total camping matrix $A$, and departure date matrix ${D_i}$. The unit camping matrices ($U_i^k$$k = 1,2,3,...$) indicate the use of campsites distributed along river banks when the $k$th trip of the trip ${p_i}$ is performed, and it is a matrix of ${n_i} \times Y$; the camping matrix ${C_i}$ indicates that the trip ${p_i}$ is utilized throughout the drifting season Campsites, we add up the matrices $U_i^k$ and, on this basis, supplement the zero matrices of $1 \times Y$ for each night camp at the time the travel plan ${p_i}$ is not implemented, so that we get a matrix of  and record Recorded as ${C_i}$ out of 25; total camp matrix $A$ represents the utilization of the entire drifting season camp, which is a sum of matrix ${C_i}(i = 1,2,3,...,25)$ and is a matrix of $179 \times Y$; departure date matrix ${D_i}$ ($i = 1,2,3,...,180$) Departure of a 180-day travel team Happening.
\par Secondly, we need to build optimization goals, and based on the requirements, our model should allow more tourist teams to drift and maximize the utilization of campsites and minimize the number of trips between tour groups. Considering the extreme state: during the 179 nights of the rafting season, all campsites live full of packages each night, which is ideal situation for rivers to accommodate the most tours, so we determine that the first optimization goal is to maximize used time of campsites in 180-day drifting season the second optimization objective is the least number of tours contact .
\par Then we consider the constraints of the model. Fully consider the use of camping sites at night for tours of acertain travel plan, the number of camps for two consecutive nights is limited by the maximum drift of the day, the occupation of any campsite throughout the drift season, and the use of all the camping sites at any one night, we construct these as the constraints of the two-objective optimization model. Considering the complexity of solving the bi-objective optimization, we weaken the second optimization objective into the constraint condition, and adjust the scheme by changing the size of its constraint value for the river management department to adopt according to the actual situation.
\par Finally, considering that the constraint conditions are random and the choice of travel days accords with the Poisson distribution, we use Java as the solution tool to get the total campmatrix and the unit camping matrix when the optimal targets met.Based on the definition of the unit matrix, we can get the number of days of travel and camping campsite per night for each travel plan. Then, according to the travel plan, we use the camping site every night to get the travel plan (oar-draft or motorboat).next, according to the time position of the unit camping matrix we can determine the start date and the end date of any travel planand.