\section{Introduction}
udhiisbsbchbas
\subsection{Background}
\noindent
There are currently about 6909 languages in the world, and the total number and geographical distribution of each language is a necessary consideration for international organizations and economic development. The so-called language population refers to the native language of the population \upcite{wiki}. In the analysis of the number of native speakers of the world language, nearly half of the population is native speakers in the following 10 languages: Mandarin (incl. Standard Chinese), Spanish, English, Hindi, Arabic, Bengali, Portuguese, Russian, Punjabi, and Japanese.

However, many people use this as their second language. The total number is not determined solely by the number of native speakers,that can determine the total number of 0.4~0.8. For a language as a second language or a third language or even more, the number of people in order and the order of the mother tongue is not the same arrangement. Therefore, when analyzing the trend of global language use, we should not only consider the number of native speakers, but also the change of the number of  second and third languages as non-native speakers. 

Over time, the increase or decrease in the total number of uses of a language is influenced by a number of factors. These factors are broadly divided into policy factors: the official language of a government or the promotion of a language, educational factors: the language of school teaching, social factors: employment pressure, cultural factors: cultural diffusion and assimilation phenomenon, demographic factors: the country's demographic changes and migration led to population migration. At the same time, due to the rapid development of the global economy, the increase of international business and transnational corporations, economic factors can drive the influence of a country's language and thus the total number of people who use language.Now the internet is popular, the world is closely related, the use of communication media and the help of mobile software, such as accurate and rapid language translation and other network factors can also affect the development of language. These factors may have an impact on the trend of language development, but not just that. 
\subsection{Restatement of the problem}
\noindent
We are required to investigate trends of global languages,and provide a multinational service company with a new international office location plan.

We understand the problem as follows:

	\begin{itemize}
		\item
We are asked to set up a model to describe the distribution of language over time based on possible influencing factors.
		\item
We should predict how the number of native speakers and total speakers will change in the next 50 years and whether the top 10 native speakers and total speakers will be replaced by another language.
		 \item
Based on the world population growth and immigration patterns for the next 50 years, we need to determine whether the geographical distribution of the language will change during this period. If so, describe the change.
		 \item 
Provide international service companies with site selection plans for new offices and consider whether the programs will be different from the perspective of long-term and short-term.
		\item 
Given the changing nature of global communications, in order to reduce the number of new international offices, we are supposed to consider additional information and give further advice based on additional information.Finally, we were also asked to write a memorandum to the relevant department. 		 	 
	\end{itemize}



\subsection{Our works}